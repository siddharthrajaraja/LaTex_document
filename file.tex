\documentclass[a4paper,12pt]{article}
\usepackage{amsmath}
\usepackage[english]{babel}
\usepackage[document]{ragged2e}
\title{\LaTeX}

\begin{document}

\centering{\huge{CHAPTER 1}} 
\linebreak

\centering{\huge{INTRODUCTION}}

\begin{quotation}
  \justify{  
            The concept of integration and differentiation is familiar to all who have studied
            elementary calculus. We know, for instance, that if $f(x)=x^2$ to the $1^{st}$ order results in
            $\int f(x)dx = \frac{1}{3} x^3 + c1$ and integrating the same function to the 
            $2^{nd}$ order results in $\int{[\int f(x)dx]dx} = \frac{1}{12} x^4 + c1x +c2$ . Similarly, 
            $\frac{d}{dx} f(x)=2x$ and $\frac{d^2}{dx^2} f(x)=2$. . However, what if we wanted to integrate our function 
            $f(x)$ to the $1/2^{th}$ order or find its $1/2^{th}$ order derivative?  How could we define our operations? Better still,
            would our results have a meaning or an application comparable to that of the familiar integer order operations? 
          }
\end{quotation}

\begin{quotation}
  \emph{1.1 The Origin of Fractional Calculus}\\   
   
  \justify{
    Fractional calculus owes its origin to a question of whether the meaning of a derivative to
    an integer order $n$ could be extended to still be valid when $n$ is not an integer. This
    question was first raised by L’Hopital on September $30^{th}$, 1695. On that day, in a letter to
    Leibniz, he posed a question about $\frac{D^n}{Dx^n}$ , Leibniz’s notation for the $n^{th}$ derivative of
    the linear function $f(x)=x$ .   L’Hopital curiously asked what the result would be if $n = \frac{1}{2}$ .
    Leibniz responded that it would be “an apparent paradox, from which one day
    useful consequences will be drawn,”[5]. \\

    Following this unprecedented discussion, the subject of fractional calculus caught the
    attention of other great mathematicians, many of whom directly or indirectly contributed
    to its development. They included Euler, Laplace, Fourier, Lacroix, Abel, Riemann and
    Liouville.
  } 

\end{quotation}

\newpage

\begin{quotation}
  
  \justify{
    
    In 1819, Lacroix became the first mathematician to publish a paper that mentioned a
    fractional derivative [3]. \\

    Starting with $y=x^m$ where m is a positive integer, Lacroix found the $n^{th}$ derivative, \\

  } 



    \begin{equation}
      \frac{d^ny}{dx^n}  = \frac{m!}{m-n!} x^{m-n}  ,  m \geq n 
      \end{equation}    
     
  \begin{flushleft}
    And using Legendre’s symbol $\Gamma$ ,  for the generalized factorial, he wrote
  \end{flushleft}    

      
    \begin{equation}
      \frac{d^ny}{dx^n}  = \frac{\Gamma(m+1)}{\Gamma(m-n+1)} x^{m-n}  
      \end{equation}    
  
  \begin{flushleft}
    Finally by letting $m=1$ and $n=\frac{1}{2}$ , he obtained
  \end{flushleft}

  \begin{equation}
    \frac{d^{\frac{1}{2}}y}{dx^{\frac{1}{2}}}  = \frac{\sqrt{x}}{\sqrt{\pi}}   
    \end{equation}    


\end{quotation}



\end{document}

